\documentclass[12pt]{article}

\usepackage[margin=0.5in]{geometry}
\usepackage{amsmath}

\begin{document}

\begin{flushright}
  Eric Murphy \\
  ECE368 \\
  Programming Assignment 3 \\
  3/19/14
\end{flushright}

\subsection*{Output and Runtime Analysis:}
Using the input r0\_flr.txt, I obtain the results:\\

Preorder: 5 3 4 1 2\\
\indent Inorder: 3 5 1 4 2\\
\indent Postorder: 3 1 2 4 5\\
\indent Width: 6.488700e+04\\
\indent Height: 9.299900e+04\\
\indent X-Coordinate 0.000000e+00\\
\indent Y-Coordinate 0.000000e+00\\
\indent Elapsed Time 0.000000e+00

\noindent \newline The amount of time that it takes from my program to run is so low due to the fact that I call malloc only once in my program, regardless of the size of the input file.
I implemented the tree structure as an array by using the index number of the rectangle to correspond to the index in the array.
The $0^{\mbox{th}}$ element was used as temporary storage throughout the program.
By using an array to implement a tree structure instead of a linked list or anything else, I am able to have $O(1)$ lookup times, making my program very fast.
Also, since I used an array as the tree structure, I was able to store the left and right values of the tree as the values of the left and right index, integers, instead of pointers.

\noindent \newline The tree traversals are done by moving from one array index to another.  Each array index holds a node that contains information involving the width, height, x and y coordinates, and left and right values.  The traversals are done by looking up the value of the left and right indicies and moving to that node in the array.

\subsection*{Space and Time Complexity:}
My space complexity can be determined by the highest number of stack frames that are in use at any given time in my function.
When doing my pre, in, and post-order ``tree'' (I implemented the tree as an array) traversals, the largest number of stack frames that are present is $log(n)$, therefore my space complexity is $O(log(n))$.
The largest number of stack frames is $log(n)$ because $log(n)$ is the height of the tree (Where $n$ is the number of nodes in the tree).

\noindent \newline My time complexity can be determined by the number of nodes that I visit on any given traversal in my program when not using loops and how many times the loops I use are iterated.
Since I have to visit every node in my array when not using loops, the time complexity of my program for the non-looping portion is $T(n) \in O(n)$.
When using loops, my program visits each node in the array only once, so the time complexity for the looping portion is $T(n) \in O(n)$.
Since both parts of my program are of $O(n)$, the total time complexity of my program is $O(n)$ (Where $n$ is the number of nodes in the tree).

\end{document}
